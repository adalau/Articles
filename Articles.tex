\documentclass{article}

% for choosing the proper font encoding
\usepackage[T1]{fontenc}
% for proper encoding
\usepackage[utf8]{inputenc}
% enhanced font Latin Modern
\usepackage{lmodern}
% multi-language support
\usepackage[english]{babel}
% for inline and display quotations.
\usepackage[autostyle]{csquotes}

% set the default 4 inches margin to be 1 inch
\usepackage{fullpage}
% for colorful link
\usepackage[ocgcolorlinks,pdfusetitle]{hyperref}
% for doi link
\usepackage{doi}
%
\usepackage{filecontents}
\begin{filecontents}{bib}
@article{ff2012svm,
  author={Fama, Eugene F. and French, Kenneth R.},
  title={{Size, value, and momentum in international stock returns}},
  journal={Journal of Financial Economics},
  year=2012,
  volume={105},
  number={3},
  pages={457-472},
  month={},
  doi={10.1016/j.jfineco.2012.05},
}
@article{asness2013vm,
  title = {Value and Momentum Everywhere},
  author = {Asness, Clifford S. and Moskowitz, Tobias J. and Pedersen, Lasse},
  year = {2013},
  journal = {Journal of Finance},
  volume = {68},
  number = {3},
  pages = {929-985},
  url = {http://EconPapers.repec.org/RePEc:bla:jfinan:v:68:y:2013:i:3:p:929-985}
}
\end{filecontents}
% for back reference in bibliography
\usepackage[style=alphabetic,citestyle=alphabetic,backend=biber,backref=true]{biblatex}
\addbibresource{bib}
\DeclareFieldFormat[inbook]{citetitle}{#1}
\DeclareFieldFormat[inbook]{title}{#1}

\begin{document}
\title{Articles}
\author{Ada Lau}
\date{\today}
\maketitle

Fama and French look at stock returns due to size, value and momentum across 4 regions (North America, Europe, Japan, and Asia Pacific). Except for Japan, value
premiums are larger for small stocks. There is no momentum anomaly in Japan. \cite{ff2012svm}. \\

Asness et. al. find consistent value and momentum premia across 4 equity markets (United States, the United Kingdom, continental Europe, and Japan), as well as across asset classes (currency, government bond and commodity). There is a strong correlation structure among value and momentum strategies across asset classes.
They find striking comovement across assets and a link to liquidity risk, and propose a Global Three-Factor Model to explain returns across asset classes. Another note is that a 50-50 combination of value momemtum perform significantly better and generate higher returns. \cite{asness2013vm}

% bibliography
\printbibliography
\end{document}

